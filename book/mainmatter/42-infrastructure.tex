\chapter{Infrastructure}

\justifying
\includegraphics{42-infrastructure.jpg}

\justifying
Software Infrastructure and Platforms are the foundations upon which our
scripts, code, and projects are founded. Infrastructure comprises the
base upon which our containers rest, and the connectivity that allows us
to communicate with them and them with each other.

\justifying
This chapter details our ability to quickly and uniformly stand up and
tear down virtual domains and networks to connect our containers and
route their workloads. We will look at some popular cloud computing
providers to prepare to explore ways we can leverage them to our
benefit.

\justifying
A Public Cloud Provider is a company that offers to host our containerized projects and virtual infrastructure
so we don't have to do it ourselves.

\justifying
The virtual resources we subscribe to will be distributed across cloud provider hardware in data centers around
the world with very little oversight or interaction from us. For example, we can choose a Region of the world
for our server instance to exist in, but we don't need to worry about which machine or rack it's in, or even
where the data center is located.

\section{Amazon Web Services (AWS)}

%\justifying
%Consider \fig{aws} which illustrates the connectivity of a basic
%project using Amazon Web Services (AWS)\index{Amazon Web Services}.

%\begin{figure}
%  \includegraphics[scale=0.45]{42-aws.png}
%  \caption{A simple Public Cloud configuration using AWS as a provider.}
%\label{aws}
%\end{figure}

\subsection{Getting Set up in AWS}

\justifying
One of the very first things you should do (after creating an account, that is) is to configure mutli-factor
authentication (MFA)\index{multi-factor authentication}.

\justifying
Amazon's AWS is one of the more prevalent cloud providers in terms of popularity and simultaneously mature and
ever-expanding feature set.

\subsection{Credentials}
\justifying
Amazon Web Services (AWS) credentials are stored in a hidden directory in your home directory called ``.aws''.
The file \textasciitilde{}/.aws/credentials should be modified to contain your AWS access\_id and secret\_key
as seen below.

\begin{mybox}{\thetcbcounter: ~/.aws/credentials}
  \lstinputlisting{code/42-infra/aws-creds.txt}
\end{mybox}

\justifying
Do not share this file with other people. Do not check this file into your GitHub repositories under any
circumstances.

\section{Google Cloud Platform (GCP)}

\justifying
Google Cloud Platform (GCP) is a suite of cloud computing services that runs on the same infrastructure that
Google uses internally for its end-user products. If the resources we allocate on GCP were a pyramid, the apex
of that pyramid would be a ``project''. A project is made up of the settings, permissions, and other metadata
that describe your application.

single: Google Cloud Platform single: GCP

\justifying
One of the very first things you should do (after creating an account, that is) is to configure two-factor
authentication\index{two-factor authentication} (2FA)\index{2FA}.

\justifying
At this point you are ready to install the gcloud\index{gcloud} software development kit (SDK) on your local
machine.

\subsection{Credentials}

\justifying
Once the gcloud SDK is installed, you are ready to set up local credentials that allow interaction between your
machine and the GCP application programming interface (API). In other words, Google hosts a server that you can
exchange commands with to configure your GCP projects from your local CLI.

\justifying
GCP credentials are stored in the directory \textasciitilde{}/.config/gcloud as a JSON file. Do not share this
file with other people. Do not check this file into your GitHub repositories under any circumstances.

\clearpage
\subsection{Directory Structure}

\justifying
Relevant folders and files related to our build pipeline are shown below. The users home directory and
workspace subdirectory is implied and removed from the diagram for clarity.

\begin{figure}[!htb]
  \input{dot/42-infra.tex}
  \caption{IaC related files and folders.}
\label{infra}
\end{figure}
